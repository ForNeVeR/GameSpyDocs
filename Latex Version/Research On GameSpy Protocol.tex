\documentclass[oneside,titlepage,a4paper]{Definition/retrospy} %book,article,report,letter


\begin{document}

\title{\Huge\textbf{Research On GameSpy Protocol}} 
\author{Arves100, Xiaojiuwo}


%\date{} %%如果没有这句,会生成时间

\maketitle  %%生成书名

\tableofcontents  %%生成目录

%\mainmatter %%表示文章的正文部分,在生成目录后将从第一页开始

\part{Research On GameSpy SDK}

\chapter{GameSpy General Construction}
\par In GameSpy SDK there are 16 modules, which constructed the GameSpy main functions.
\section{GameSpy SDK Module}
\begin{itemize}
	\item GameSpy Presence Servers
	\begin{itemize}
		\item GameSpy Presence Connection Manager
		\item GameSpy Presence Search Player
	\end{itemize}
	\item Nat Negotiation
	\item Master Server: Query Report 2
	\item Master Server: Server Browser
	\item Master Server: Available Check
	\item Game Patching
	\item Game Tracking
	\item Master Server Patching: Downloading files from FilePlanet
	\item Peer SDK
	\item Game Statitics
	\item Chat Server
\end{itemize}

\chapter{GameSpy Presence SDK}
\section{GameSpy Presence Connection Manager}
\par GameSpy Presence Servers contain two server, GameSpy Presence Connection Manager (GPCM) and GameSpy Presence Search Player (GPSP). GPCM is a server that handle login request and response with corresponding user infomation stored on GameSpy. GPSP is a server that handle search request for user.
\subsection{Server IP and Ports}
Table \ref{IP and Ports for GameSpy Presence Servers} are the GPCM and GPSP IP and Ports that client/game connect to.
\begin{table}[H]
	\centering
	\begin{tabular}{|c|c|}
		\hline 
		\textbf{IP}&\textbf{Port}  \\ 
		\hline 
		gpcm.gamespy.com&29900 \\ 		
		\hline 
	 	gpsp.gamespy.com&29901 \\
	 	\hline 
	\end{tabular} 
\caption{IP and Ports for GameSpy Presence Servers}
\label{IP and Ports for GameSpy Presence Servers}

\end{table}

\subsection{Database Key Field}
These keys is that GameSpy Presence SDK using to find a user in their database. Keys are shown in Table \ref{Key Field}.

\begin{table}[H]
	\centering
	\begin{tabular}{|c|>{\centering\arraybackslash}p{8cm}|}
		\hline 
		\textbf{Keys}& \textbf{Description}  \\ 
		\hline 
		User & An user contains the Email and the password, but contains multiple profiles \\ 		
		\hline 
		ProfileID & The profile contains the name, surname, birth date and all the rest user info, including
		an unique nickname used to identify the profile and a generic nickname used to show for example in
		games \\
		\hline 
	\end{tabular} 
	\caption{Key Field}
	\label{Key Field}	
\end{table}

\subsection{Protocol Descriptions}

In this part, we show the protocol detail in GameSpy Presence SDK.
\subsubsection{The String Pattern}
We first introduce the pattern of the string, which is using to make up a request.
This kind of string is represent a value in a request sends by the client as Table \ref{Value string}.

\begin{table}[H]
	\centering
	\begin{tabular}{|c|c|}
		\hline 
		\textbf{String}&\textbf{Description}  \\ 
		\hline 
	$ \backslash \langle content \rangle \backslash $& The value is $ \langle content \rangle $  \\ 
		\hline 
	\end{tabular} 
\caption{Value string}
\label{Value string}
\end{table}

This kind of string is represent a command in a request sends by the client as Table \ref{Command string}. The command will end with $ \backslash \backslash $ or $ \backslash $ depends on whether run at the server-side or client-side.


\begin{table}[H]
	\centering
	\begin{tabular}{|c|c|}
		\hline 
		\textbf{String}&\textbf{Description}  \\ 
		\hline 
		$ \backslash command \backslash\backslash $& This is a command \\ 		
		\hline 
		$ \backslash error \backslash \backslash $ & Error command \\
		 \hline
		 $\backslash lc \backslash$& Login command\\
		 \hline
	\end{tabular} 
	\caption{Command string}
	\label{Command string}
\end{table}

This kind of string is represent a parameter in a request sends by the client \ref{Parameter string}. GameSpy uses the combination of the parameter to search the string with value, and sends the data back to client use this kind of parameter string.

\begin{table}[H]
	\centering
	\begin{tabular}{|c|c|}
		\hline 
		\textbf{String}&\textbf{Description}  \\ 
		\hline 
		$ \backslash id \backslash 1 \backslash $& This is a parameter string the value of $ id $ is $ 1 $ \\ 		
		\hline 
		$ \backslash profileid \backslash 007 \backslash $ & This is a parameter string the value of $ profileid $ is $ 007 $ \\
		\hline
	\end{tabular} 
	\caption{Parameter string}
	\label{Parameter string}
\end{table}

Error response string for (GPCM, GPSP):
\begin{equation}
\begin{split}
\backslash error \backslash\backslash err \backslash \langle error code \rangle \backslash fatal\backslash\backslash errmsg \backslash \langle error message \rangle \backslash id\backslash 1 \backslash final \backslash
\end{split}	
\end{equation}
\subsubsection{Login Phase}
\myparagraph{Client Login Request}
There are three ways of login:
\begin{itemize}
	\item AuthToken: Logging using an alphanumeric string that rapresents an user
	\item 	UniqueNick: Logging using a nickname that is unique from all the players
	\item User: Logging with the nickname and the password
\end{itemize}

The full login request string:
\begin{equation}\label{Chanllenge string}
\begin{split}
	&\backslash login \backslash challenge \backslash \langle challenge \rangle \backslash authtoken \backslash \langle authtoken \rangle \\& \backslash uniquenick \backslash \langle uniquenick \rangle \backslash user \backslash \langle user \rangle 
	\backslash userid \backslash \langle userid \rangle \\& \backslash profileid \backslash \langle profileid \rangle \backslash partnerid \backslash \langle partnerid \rangle \backslash response \backslash \langle response \rangle \\&
	 \backslash firewall \backslash 1 \backslash port \backslash \langle port \rangle \backslash productid \backslash  \langle productid \rangle \\& \backslash gamename \backslash \langle gamename \rangle \backslash namespaceid \backslash \langle namespaceid \rangle \\& \backslash  sdkrevision \backslash \langle sdkrevision \rangle \backslash quiet \backslash \langle quiet \rangle \backslash id \backslash 1 \backslash final \backslash
\end{split}
\end{equation}
The value $ \langle challenge \rangle $ for $ \backslash challenge \backslash $ in \ref{Chanllenge string} is a 10 byte alphanumeric string.

The following Table \ref{Login parameter string} is a description of string used in login request, GameSpy can use these string to find value in database.
\begin{table}[H]
	\centering
%\begin{tabular}{|>{\centering\arraybackslash}m{2cm}|>{\centering\arraybackslash}m{7cm}|>{\centering\arraybackslash}m{1cm}|}

\begin{tabular}{A}
		\hline
		\textbf{Keys} & \textbf{Description} & \textbf{Type}
			                                                                          \\ \hline
		 login& The login command which use to identify the login request of client&\\ \hline
		 challenge  & The user challenge used to verify the authenticity of the client     &                                                                                                                                     \\ \hline
		 authtoken  & The token used to login (represent of an user)        &                                                                                                                                                    \\ \hline
		uniquenick  & The unique nickname used to login       &                                                                                                                                                                  \\ \hline
		   user     & The users account (format is NICKNAME@EMAIL)           &                                                                                                                                                   \\ \hline
		  userid    & Send the userid (for example when you disconnect you will keep this)              &                                                                                                                        \\ \hline
		 profileid  & Send the profileid (for example when you disconnect you will keep this)           &                                                                                                                        \\ \hline
		 partnerid  & This ID is used to identify a backend service logged with gamespy.(Nintendo WIFI Connection will identify his partner as 11, which means that for gamespy, you are logging from a third party connection) &\\ \hline
		 response   & The client challenge used to verify the authenticity of the client     &                                                                                                                                   \\ \hline
		 firewall   & If this option is set to 1, then you are connecting under a firewall/limited connection &\\
		 \hline
		 port& The peer port (used for p2p stuff)            &                                                              \\ \hline
		 productid  & An ID that identify the game you're using            &                                                                                                                                                     \\ \hline
		 gamename   & A string that rapresents the game that you're using, used also for several activities like peerchat server identification             &                                                                    \\ \hline
		
		namespaceid & ?   &                                                                                                                                                                                                      \\ \hline
		sdkrevision & The version of the SDK you're using& \\ \hline
		   quiet    & ? Maybe indicate invisible login which can not been seen at friends list &\\ \hline
		   id& The value is 1&\\ \hline
		   final & Message end&\\ \hline
	\end{tabular} 
	\caption{Login parameter string}
	\label{Login parameter string}
\end{table}
\paragraph{Login Response From Server}\mbox{}\\

This response string \ref{Login response string1}, \ref{Login response string2} is send by the server when a connection is accepted, and followed by a challenge\ref{Chanllenge string}, which verifies the server that client connect to.
\par There are two kinds of login response string: 
\begin{equation}\label{Login response string1}
\begin{split}
&\backslash lc \backslash 1 \backslash challenge \backslash \langle challenge \rangle \backslash nur \\&\backslash userid \backslash \langle userid \rangle \backslash profileid \backslash \langle profileid \rangle \backslash final \backslash
\end{split}	
\end{equation}

\begin{table}[H]
	\centering
	\begin{tabular}{|>{\centering\arraybackslash}m{2cm}|>{\centering\arraybackslash}m{7cm}|>{\centering\arraybackslash}m{1cm}|}

		\hline 
		\textbf{Keys}&\textbf{Description}&\textbf{Type}  \\ 
		\hline 
		challenge & The challenge string sended by GameSpy Presence server&  \\ 		
		\hline 
		nur & ? &\\
		\hline 
 userid&The userID of the profile &\\	\hline 
 profileid&The profileID &\\	\hline 
 final& &\\	\hline 
	\end{tabular} 
	\caption{The first type login response}
	\label{The first type login response}	
\end{table}

\begin{equation}\label{Login response string2}
\begin{split}
&\backslash lc \backslash 2 \backslash sesskey \backslash \langle sesskey \rangle  \backslash userid \backslash \langle userid \rangle \backslash profileid \backslash \langle profileid \rangle \\& \backslash uniquenick \backslash \langle uniquenick \rangle \backslash lt \backslash \langle lt \rangle \backslash proof \backslash \langle proof \rangle \backslash final \backslash
\end{split}	
\end{equation}

\begin{table}[H]
	\centering
	\begin{tabular}{|>{\centering\arraybackslash}m{2cm}|>{\centering\arraybackslash}m{7cm}|>{\centering\arraybackslash}m{1cm}|}
		\hline 
		\textbf{Keys}& \textbf{Description}&\textbf{Type}  \\ 
		\hline 
		sesskey & The session key, which is a integer rapresentating the client connection& \\ 		
		\hline 
		userid & The userID of the profile& \\
		\hline 
		profileid&The profileID &\\	\hline 
		uniquenick&The logged in unique nick &\\	\hline 
		lt& The login ticket, unknown usage&\\\hline
		proof& The proof is something similar to the response but it vary&\\\hline
		final& &\\
	\hline 
	\end{tabular} 
	\caption{The second type login response}
	\label{The second type login response}
\end{table}
Proof in \ref*{The second type login response} generation: $ md5(password)||48 spaces $
The user could be AuthToken or the User/UniqueNick (with the extra PartnerID).
server challenge that we received before.
the client challenge that was generated before.




\subsubsection{User Creation}
This commmand \ref{Create user command} is used to create a user in GameSpy.
\begin{equation}\label{Create user command}
\begin{split}
&\backslash newuser \backslash email \backslash \langle email \rangle \backslash nick \backslash \langle nick \rangle \\& \backslash passwordenc \backslash \langle passwordenc \rangle 
\backslash productid \backslash \langle productid \rangle \\& \backslash gamename \backslash \langle gamename \rangle \backslash uniquenick \backslash \langle uniquenick \rangle \\& \backslash cdkeyenc \backslash \langle cdkeyenc \rangle \backslash partnerid \backslash \langle partnerid \rangle \backslash id \backslash 1 \backslash final \backslash
\end{split}	
\end{equation}
The description of each parameter string is shown in Table \ref{User creation string}.
\begin{table}[H]
	\centering
	\begin{tabular}{|>{\centering\arraybackslash}m{2cm}|>{\centering\arraybackslash}m{7cm}|>{\centering\arraybackslash}m{1cm}|}
		\hline
		  \textbf{String}    & \textbf{Description} &\textbf{Type}                                                                 \\ \hline
		   email    & The email used to create                                                    & \\ \hline
		   nick     & The nickname that will be created                                            & \\ \hline
		passwordenc & The encoded password (password XOR with Gamespy seed and the Base64 encoded)  &\\ \hline
		 productid  & An ID that identify the game you're using                                     &\\ \hline
		 gamename   & A string that rapresents the game that you're using, used also for several    &\\ \hline
		namespaceid & ?Unknown                                                                      &\\ \hline
		uniquenick  & Uniquenick that will be created                                               &\\ \hline
		 cdkeyenc   & The encrypted CDkey, encrypted method is the same as the passwordenc          &\\ \hline
		 partnerid  & This ID is used to identify a backend service logged with gamespy             &\\ \hline
		    id      & The value of id is 1                                                          &\\ \hline
		   final    & Message end                                                                   &\\ \hline
	\end{tabular}
\caption{User creation string}
\label{User creation string}
\end{table}

\section{GameSpy Presence Search Player}

\subsection{Search User}
This is the request that client sends to server:
\begin{equation}\label{Search User Request}
\begin{split}
 \backslash search\backslash\backslash sesskey \backslash<sesskey>\backslash profileid \backslash <profileid> \\ \backslash namespaceid\backslash <namespaceid>  \backslash partnerid\backslash <partnerid>\\ \backslash nick \backslash <nick> \backslash uniquenick \backslash <uniquenick> \\ \backslash email \backslash <email> \backslash gamename \backslash <gamename> \backslash final \backslash
\end{split}
\end{equation}

This is the response that server sends to client:
\begin{equation}
	\begin{split}
	\backslash bsr \backslash <profileid> \backslash nick \backslash <nick>	\backslash uniquenick \backslash <uniquenick> \\
	\backslash namespaceid \backslash <namespaceid>\backslash firstname \backslash <firstname> \\ 
	\backslash lastname \backslash <lastname>\backslash email \backslash <email> \\
	\backslash bsrdone \backslash <gamespy enc determinator> \backslash final \backslash
	\end{split}
\end{equation}



\part{RetroSpy System Architecture}
\section{GameSpy Library}
	\subsection{Networks}
	 There are two different servers in RetroSpy; one is TCP another is UDP.  TCP and UDP work differently so the implementation will be different. We show the different implementing in \ref{Tcp} and \ref{Udp}.
		\subsubsection{TCP}\label{Tcp}
		    TcpServer class is only for making the connection and listening for connections. TcpStream is for receiving and sending the message.
		\subsubsection{UDP}\label{Udp}
		  UdpServer class does not need a server to handle connection and listen for connection, every client can be a server, and every server is a client. So this class has both receiving and sending functions.
\chapter{introduction}

\chapter{conclusion}

\end{document}
